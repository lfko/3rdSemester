\section{Aktuelle Entwicklungen}

Moderne Gefängnisse unterscheiden sich mittlerweile deutlich von ihren historischen Vorgängern. Zwar stehen immer noch die 5 Prinzipien \textit{Rehabilitation, Abschreckung, Wegsperrung, Vergeltung} und \textit{Denunziation (i.S.v. öffentlich Anprangern)} im Vorderund. Dennoch sollen und müssen zeitgenössische Gefängisse mittlerweile auch andere Aufgaben abdecken, wie: 
\begin{itemize}
	\item der Sicherung des Insassen vor der Gesellschaft und \textit{vice versa}
	\item Umsetzung der 5 Prinzipien (s.o.)
	\item Menschenwürdige Unterbringung
	\item Ermöglichung von Weiterbildung und entsprechendem Freizeitausgleich
	\item Wahrnemung des Gefängnisses als ökonomischer Faktor
\end{itemize}

Zwei Beispiele moderner Gefängnissarchitekturen sollen folgend als Beispiel aufgeführt werden.

\subsection{ADMAX Florence}
Das \textbf{ADMAX Florence} ist ein sogenanntes \textit{super-maximum security} Gefängnis und gilt als eines der sichersten der Welt. Dort untergebrachte Insassen werden in völliger Isolation gehalten; die Haftbedingungen selber sind unbekannt, sollen aber den Mindestanforderungen entsprechen. Diese Art von Gefängnis verfolgt den bereits genannten Ansatz des \textit{Pennsylvania System} und dient primär dem Wegsperren der Insassen und legt keinen Wert auf Rehabilitation.

\subsection{Halden Fengsel}
Das \textbf{Halden Fengsel} verfolgt einen gänzlich anderen Ansatz, als das zuvor genannten \textbf{ADMAX Florence}. In seiner Art immer noch ein Gefängnis, scheint es von außen eher wie ein Therapiezentrum. Es ist so offen wie notwendig, so geschlossen wie nötig konzipiert und wirkt in seiner Gesamtkomposition eher wie ein "Dorf" denn ein Gefängnis. Den Insassen soll somit verdeutlicht werden, dass sie immer noch ein Teil der Gesellschaft, der Gemeinschaft sind. Entsprechend wird auch versucht, die Gefangenen zu rehabilitieren, sie aktiv bei diesem Prozess zu unterstützen. 
Helle Farben und entsprechende Raumkonzepte dienen als passive, unterschwellige Unterstützung bei dieser Thematik.
