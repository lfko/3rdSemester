\section{Geschichte der Strafsprechung}
In diesem Abschnitt soll ein kurzer, geschichtlicher Abriss über die Entwicklung der Strafsprechung gegeben werden und vor allem darauf eingehen, wie der Begriff der \textit{Strafe} in verschiedenen Zeiten verstanden wurde und was damit verbunden war.

\subsection{Antike - frühes Mittelalter}
Das Prinzip, seine Schuld im Sinne einer Buße abzusitzen, war damals gänzlich unbekannt. Die Bestrafung für eine begangene Misstat erfolgte nach dem Ansatz \textit{quid pro quo} und sah so bspw. empfindliche Leibesstrafen vor, angefangen von Folter und Misshandlung bis hinzu Verstümmelung oder sogar der Todesstrafe, mitunter auch Verbannung aus der Gemeinschaft und Ächtung als Vogelfreier.
Bestimmte Delikte, wie z.B. Schulden, sah schon ein Einsperren vor (sog. \textit{Schuldner-Türme}), waren jedoch selten. Eine weitere Art der Verbannung war auch die Versklavung, z.B. als Ruderer auf einer Galeere.