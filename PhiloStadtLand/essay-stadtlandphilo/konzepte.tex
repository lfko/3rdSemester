%\subsection{Einleitung}
\section{Gefängniskonzepte/-architekturen}
Ab dem 18. Jahrhundert wurde begonnen, das Gefängnis als solches konzeptionell zu überdenken und zu entwerfen, vor allem ausgelöst durch das Aufkommen der Aufklärung und tendenziell humanistischer Betrachtungsweisen des Menschen und der Gesellschaft. Vordenker dieser Bewegung waren, u.a., Jeremy Bentham (1748 - 1832) - Philosoph und Jurist - sowie John Howard (1726 - 1790). Beide beschäftigten sich mit dem Zustand des Gefängnissystemes und der Jurisdiktion und ersannen einflussreiche Veränderungen. Ersteres entwarf das sogennante \textit{Panopticon}, auf welches im nächsten Abschnitt eingegangen werden soll. Es gilt als das erste Konzept-Gefängnis in der Geschichte.

\subsection{Panopticon}
Mit dem \textbf{Panopticon} (von griech. \textit{pan 'alles'} und \textit{optico 'zum Sehen gehörend'}) unternahm Bentham den Versuch, ein Gefängnis zu entwerfen, welches allzeitige Be- und Überwachung der Gefangenen ermöglichen sollte. Erreicht werden sollte dies durch eine Rundbauweise, in welcher die Zellen kreisförmig um eine zentralen Wachturm herum angeordnet sind, von welchem aus die Wärter Einsicht haben können. \\
Die Idee hinter dieser Architektur lässt sich als \textit{vollkommende Überwachung} beschreiben, da - bedingt durch diese bauliche Eigenheit - der Insasse nie sicher sein kann, nicht beobachtet zu werden. Im Umkehrschluss wird somit ein sich selbsterziehendes System oktroyiert, da der Gefangene, auch um seiner selbst Willen, entsprechendes Betragen an den Tag legen sollte, um nicht weitere Repressalien zu erfahren. Das Eingesperrtsein wird somit zur ursächlichen Straferfahrung. als Reformationsmaßnahme.
\subsection{Pennsylvania System}
Am 25.10.1829 erfolgte die Öffnung des \textbf{Eastern State Penitentiary} des Architekten \textbf{John Haviland} und galt als Prototyp des sog. \textit{Pennsylvania System}, da es das erste Gefängnis dieser Art war.
Tatsächlich lassen sich auch hier Ansätz des \textbf{Panopticons} wiederfinden, wenngleich ein anderer sozialer Ansatz als Grundlage implementiert wurde: \textit{Separation/Isolation}. \\
Die Gefangenen wurden in größeren Zellenblöcken inhaftiert, mussten jedoch selber in absoluter Stille verbleiben, d.h. durften weder mit sich selbst noch untereinander kommunizieren und ihre Inhaftierung in Buße und stillem Gebet verbringen, um somit eine Retribution zu erfahren und für ihre Schuld zu sühnen. Daraus ergab sich allerdings das Problem, dass die Gefangenen durch diese omnipräsente Situation der Isolation zunehmend zu Psychosen neigten und es mitunter zu erhöhten Zahlen an geistigen Krankheiten und Selbstmorden kam. 

\subsection{Auburn System}
Das \textbf{Auburn System} stellte kein direktes Gegenprinzip zum \textbf{Pennsylvania System} dar - so setzte es auch auf die Isolation der Gefangenen -, sah in seinem Konzept jedoch vor, dass die Insassen einen Teil ihrer Zeit pro Tag mit Arbeit zu verbringen hatten. Dies erfüllte einerseits den Zweck der Wiedergutmachung und Buße, schärfte auf der anderen Seite aber auch wieder den Sinn für ehrliche und gerechte Arbeit, Lohn, Eigentum, sittsames Benehmen in der Gesellschaft. Für die Betreiber der Gefängnisse wiederrum ergab sich dadurch die Möglichkeit, an billige Arbeitskräfte zu gelangen, welche dann gegebenenfalls als "Lohnarbeiter" für andere Arbeitgeber attraktiv wurden.
Dieses System bewährte sich in den kommenden Jahren und wurde auch auf andere Kontinente, in andere Länder "exportiert" und dort entsprechend adaptiert.